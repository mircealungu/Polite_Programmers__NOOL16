%!TEX root=../paper.tex

\section{Illustrating the Polite Syntax}

To experiment with sentence case identifiers, we started from the grammar of Smalltalk and we modified all the rules that involve identifiers to allow spaces in identifier names. Once this was done, we discovered several other grammar modifications which we considerd desirable. We named the resulting language Polite Smalltalk or for short, Polite. Others in the past have also started from Smalltalk and provided deltas to augment its syntax \cite{Born87}.

In this section we discuss those modifications to the original  Smalltalk language that were performed to obtain Polite.

% Polite, the language that we propose here, started as an experiment in evolving the syntax of Smalltalk to using sentence case identifiers. We thus started from the Smalltalk grammar that we had implemented as a parsing expression grammar and iteratively modified it. 

% It is derived from Smalltalk to which we add sentence case identifiers, first class functions, and the concept of a program. 

% The syntax of Polite is basically that of Smalltalk-80, to which we bring several modifications. The reader can thus assume the Smalltalk syntax, and we will discuss the modifications in this section. 

\newcommand{\comma}{{`,' }}
\newcommand{\plus}{{`+' }}
\newcommand{\code}[1]{{\texttt{#1}}}

\subsection{Sentence Case Identifiers}
In Smalltalk, to send to an object a unary message, one simply separates the two with a space, e.g.

\begin{minted}{text} 
politeHero rechargeEnergy.
\end{minted}

Keyword messages are similar but they are separated by their arguments with a column, e.g.

\begin{minted}{text} 
politeHero fightWith: anEnemy.
\end{minted}


If we want to allow spaces in the name of identifiers, we cannot use space to separate the object and the message. In Polite we introduce a comma to separate the name of an agent from the message that is receives, as one would do in natural language when instructing an agent to do something (e.g. {\em ``Alfred, get the Batwing ready''}). All the characters in an identifier including the spaces are relevant for its identity. The previous code snippets thus become: 

\begin{minted}{text} 
polite hero, recharge energy. 
polite hero, fight with: an enemy.
\end{minted}

For the seasoned Smalltalkers, using \comma in this way might be offensive since it is traditionally used as the concatenation operator for strings and other collections. However, such a sentiment is misplaced since \comma is a simple message implemented in the Collection class. In Polite, we overload the \plus operator to take over the responsibilities of \comma.


\subsection{Polite Programs}
The Smalltalk grammar does not provide productions for programs or classes, since the programmer grows a program by compiling a method at a time in the Smalltalk IDE. 

To be able to write programs independent of the Smalltalk IDE we introduce a new grammar rule for a program: a sequence of class and global method definitions, followed by code to be executed. The following code snippet presents a simple program with one class definition and a few lines of code: 
\vspace{0.5em}

\inputminted[bgcolor=lbcolor,linenos]{text}{polite-hero.polite}

\vspace{0.5em}
In the code snippet we see several other features of the language: 

\begin{itemize}

	\item A class declaration (line 1) has the syntax of a message sent to the superclass for creating the subclass. 

	\item Indentation based scoping is used for defining the body of a class and the body of a method. 

	\item Class, program, and method local variable declarations (lines~2 and 17) are enumerated between pipes and must be separated by~\comma (line 2)

\end{itemize}


\subsection {First Class Functions}
Polite allows the definition of first class functions as opposed to Smalltalk where a method must always be part of a class. These functions are global to the program. 

One of the benefits of this, is the possibility of implementing new control structures like the one used in lines 10--14 from the example program: 

\begin{minted}{text}
while neither: [self, is dead] 
	  nor: [an enemy, is dead] 
	   do: [self, fight with: an enemy] 
\end{minted}

The possibility of defining such ad-hoc control structures, that other languages like Grace or Scala also provide, can be used to improve the readability of program text. This is especially beneficial for several of the conditional structures of Smalltalk, since a study by Stefik shows that they are not very intuitive for newcomers to the language \cite{Stef13}. To illustrate the difference in readability, compare the previous code with the equivalent traditional Smalltalk: 

\begin{minted}{text}
((self isDead not) and: [anEnemy isDead not]) 
	whileTrue: [self fightWith: anEnemy]
\end{minted}




% The highlighted line of code will be hard to read for beginners [citation about control structures in St being hard in general for beginners] and could even be error prone for advanced developers.  shows the usefulness of having first class functions. Especially since the control structures of Smalltalk turn out to puzzle beginners [citation needed]. 
